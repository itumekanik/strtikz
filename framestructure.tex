\pgfkeys{
 /framestructure/.is family, /framestructure,
  default/.style = {number of storys = 5,
  number of bays = 3,
  story height = 1cm,
  bay width = 2cm,
  startX = 0cm,
  startY = 0cm,
  frame line thickness = 1pt,
  support width = 0.3cm,
  support height = 0.15cm,
  support line thickness = 1.5pt,
  show supports = 1,
  isolator width = 0.3cm,
  isolator thickness = 0.2cm,
  isolator line thickness = 1.5pt,
  foundation thickness = 0.5cm,
  foundation side width = 1cm,
  mass radius = 2pt,
  show mass = 1,
  dof floor = 1,
  dof column = 2,
  arrow length ratio = 0.4,
  dof x rotation = 0,
  dof y rotation = 0,
  dof r rotation = 0,
  dof offset ratio = 0.075,
  rotation dof start angle = 120,
  rotation dof end angle = 280,
  show dof = 1,
  show axes = 1,
  subfloor number=2,},
  number of storys/.store in = \storynumber,
  number of bays/.store in = \baynumber,
  story height/.store in = \storyheight,
  bay width/.store in = \baywidth,
  startX/.store in = \startx,
  startY/.store in = \starty,
  frame line thickness/.store in = \framelinet,
  support width/.store in = \supportwidth,
  support height/.store in = \supportheight,
  support line thickness/.store in = \baselinet,
  show supports/.store in = \showsupports,
  isolator width/.store in = \isolationwidth,
  isolator thickness/.store in = \isolationdepth,
  isolator line thickness/.store in = \isolinet,
  foundation thickness/.store in = \foundationdepth,
  foundation side width/.store in = \foundsidew,
  mass radius/.store in = \massrad,
  show mass/.store in = \showmass,
  dof floor/.store in = \doflocfloor,
  dof column/.store in = \dofloccolumn,
  arrow length ratio/.store in = \arrowlenratio,
  dof offset ratio/.store in = \dofoffsetratio,
  dof x rotation/.store in = \dofxrotation,
  dof y rotation/.store in = \dofyrotation,
  dof r rotation/.store in = \dofrrotation,
  rotation dof start angle/.store in = \rotdofstartangle,
  rotation dof end angle/.store in = \rotdofendangle,
  show dof/.store in = \showdof,
  show axes/.store in = \showaxes,
  subfloor number/.store in=\subfloors,}

\newcommand{\framestructure}[1][]{
\pgfkeys{/framestructure, default, #1}
\tikzmath{
int \storynumber, \baynumber, \columnnumber, \levelnumber, \nlevmo, \ncolmo, \iii, \j, \showsupports;
real \storyheight, \baywidth, \startx, \starty, \xx, \y;
real \supportwidth, \supportheight, \isolationwidth, \isolationdepth, \foundationdepth, \massrad;
real \axissp, \framelinet, \baselinet, \isolinet;
real \rigbasestartx, \rigbaseendx, \isoboty, \isotopy;
real \foundboty, \foundtopy, \foundstartx, \foundendx;
int \doflocfloor, \dofloch, \dofloccolumn, \showaxes, \showdof, \showmass, \showsupports;
real \arrowlenratio, \minlen, \dofxx, \dofyy, \arrlen, \arrrad;
real \dofxrotation, \dofyrotation, \dofrrotation;
real \rotdofstartangle, \rotdofendangle;
int \subfloors;
real \basewallstartx, \basewallstarty, \basewallendx, \basewallendy, \buildingwidth, \basewalldepth;
%Conversion all units to the unit pt
\storyheight = \storyheight;
\baywidth = \baywidth;
\startx = \startx;
\starty = \starty;
\supportwidth = \supportwidth;
\supportheight = \supportheight;
\isolationwidth = \isolationwidth;
\isolationdepth = \isolationdepth;
\foundationdepth = \foundationdepth;
\framelinet = \framelinet;
\baselinet = \baselinet;
\isolinet = \isolinet;
\massrad = \massrad;
\foundsidew = \foundsidew;
%End conversion, All values are now in pt units.
\axissp = 0.2cm;
%\framelinet = 1.0pt;
%\baselinet = 1.5pt;
%\isolinet = 3pt;
%\foundationdepth =1cm;
\columnnumber = \baynumber+1; %number of columns
\levelnumber = \storynumber+1; %number of levels
if \storynumber>1 then {\nlevmo = \levelnumber-1;} else {\nlevmo=2;};
if \baynumber>1 then {\ncolmo = \columnnumber-1;} else {\ncolmo=2;};
for \iii in {1,...,{\levelnumber}}{
\y{\iii} = (\iii-1)*\storyheight;
for \j in {1,...,{\columnnumber}}{
\x{\j} = (\j-1)*\baywidth;
};
};
\rigbasestartx = \x1-\supportwidth;
\rigbaseendx = \x{\columnnumber}+\supportwidth;
\isoboty = -\supportheight-\baselinet/2-\isolationdepth;
\isotopy = -\supportheight-\baselinet/2;
\foundboty = -\supportheight-\baselinet-\isolationdepth-\foundationdepth;
\foundtopy = -\supportheight-\baselinet-\isolationdepth;
\foundstartx = \x1-\foundsidew;
\foundendx = \x{\columnnumber}+\foundsidew;
\basewallstartx=-\supportwidth;
\buildingwidth=\baynumber*\baywidth;
\basewalldepth=\subfloors*\storyheight;
\basewallstarty=\basewalldepth;
\basewallendx=\buildingwidth+\supportwidth;
\basewallendy=\basewallstarty;
%%%DOFs%%%%
if greater(\doflocfloor,\storynumber) then {\doflocfloor=0;} else{};
if greater(\dofloccolumn,\columnnumber) then {\dofloccolumn=1;} else{};
%\doflocfloor = 2;
%\dofloccolumn = 2;
%\arrowlenratio = 0.4;
%\dofoffsetratio = 0.075;
\dofloch = 1+\doflocfloor;
\minlen = min(\storyheight,\baywidth);
\dofxx = \x{\dofloccolumn}+\dofoffsetratio*\minlen;
\dofyy = \y{\dofloch}+\dofoffsetratio*\minlen;
\arrlen = \arrowlenratio*\minlen;
\arrrad = \arrlen*0.8;
%%%%%Axes%%%%%
\axeslenX = 0.5cm;
\axeslenY = 0.5cm;
\Xaxesstarty = \y{1};
\Yaxesstartx = \x{1};
\Yaxesstarty = \y{\levelnumber}+\axissp;
if equal(\showsupports,0) then {\Xaxesstartx = \x{\columnnumber}+\axissp;}         else {\Xaxesstartx = \x{\columnnumber}+\axissp+\supportwidth/2;};
if equal(\showsupports,1) then {\Xaxesstartx = \x{\columnnumber}+\axissp+\supportwidth/2;} else {\Xaxesstartx = \x{\columnnumber}+\axissp+\supportwidth;};
%\rotdofstartangle = 100;
%\rotdofendangle = 100;
%%%%%%%%%%%%%%%%%%%%%%%%
}
\begin{scope}[x=1pt, y=1pt, xshift=\startx, yshift=\starty, rotate=0]; % Drawing everything in pt units
%\draw (0,0)-- (0.5cm,2cm);
\draw [line width = \framelinet] (\x{1},\y{1}) -- (\x{1}, \y{\levelnumber}) -- (\x{\columnnumber}, \y{\levelnumber}) -- (\x{\columnnumber}, \y{1});
\foreach \iii in {2,...,{\nlevmo}}
{\draw [line width = \framelinet] (\x{1},\y{\iii}) -- (\x{\columnnumber},\y{\iii});}

\foreach \j in {2,...,{\ncolmo}}
{\draw [line width = \framelinet] (\x{\j},\y{1}) -- (\x{\j},\y{\levelnumber});}


\ifthenelse{\showmass=1}{
\ifthenelse{\showsupports=5}{
	\foreach \iii in {\useeval{2+\subfloors},...,{\levelnumber}}
	{\foreach \j in {1,...,{\columnnumber}}{
	\shade[ball color=black] (\x{\j},\y{\iii}) circle (\massrad);
	}}
	\foreach \iii in {2,...,\useeval{\subfloors+1}}
	{\foreach \j in {2,...,\useeval{\columnnumber-1}}{
	\shade[ball color=black] (\x{\j},\y{\iii}) circle (\massrad);
	}}
}
{
	\foreach \iii in {2,...,{\levelnumber}}
	{\foreach \j in {1,...,{\columnnumber}}{
	\shade[ball color=black] (\x{\j},\y{\iii}) circle (\massrad);
	}}
}
}{}

\ifthenelse{\showsupports=0}{
}{}

\ifthenelse{\showsupports=1}{
\foreach \j in {1,...,{\columnnumber}}
{
\fill [gray] (\x{\j},\y{1}) +({-\supportwidth/2},-\supportheight) rectangle +({\supportwidth/2},\y{1});
\draw [line width = \baselinet] (\x{\j},\y{1}) +({-\supportwidth/2},\y{1}) -- +({\supportwidth/2},\y{1});
}
}{}

\ifthenelse{\showsupports=2}{
\fill [gray] (\rigbasestartx, 0) rectangle (\rigbaseendx,-\supportheight);
\draw [line width = \baselinet] (\rigbasestartx, 0) -- (\rigbaseendx,0);
}{}

\ifthenelse{\showsupports=3}{
\foreach \j in {1,...,{\columnnumber}}
{\fill [fill=gray, draw=black, line width = \baselinet] (\rigbasestartx, 0) rectangle (\rigbaseendx,-\supportheight);}
}{}

\ifthenelse{\showsupports=4}{
\fill [fill=gray, draw=black, line width = \baselinet] (\rigbasestartx, 0) rectangle (\rigbaseendx,-\supportheight);

\foreach \j in {1,...,{\columnnumber}}
{\filldraw [fill = white!20!black, draw=black, line width=\isolinet]
(\x{\j},\y{1}) +(-\isolationwidth/2,\isoboty) rectangle +(\isolationwidth/2,\isotopy);}

\fill [fill=gray]
(\foundstartx, \foundboty) rectangle (\foundendx,\foundtopy);
\draw [line width = \baselinet] (\foundstartx, \foundtopy) -- (\foundendx, \foundtopy);
}{}

\ifthenelse{\showsupports=5}{
\filldraw[line width = \baselinet, fill=gray] (\basewallstartx,\basewallstarty) -- ++(\supportwidth,0) -- ++(0,-\basewalldepth)
-- ++(\buildingwidth,0) -- ++(0,\basewalldepth) -- ++(\supportwidth,0) -- ++(0,-\basewalldepth-\supportheight)
-- (\basewallstartx,-\supportheight) -- cycle;
}

\ifthenelse{\showaxes=1}{
\draw [{->}] (\Xaxesstartx,\Xaxesstarty) -- +(\axeslenX,0) node[above]{$X$};
\draw [{->}] (\Yaxesstartx,\Yaxesstarty) -- +(0,\axeslenY) node[right]{$Y$};
}{}

\ifthenelse{\showdof=1}{
\dofs[startx = \dofxx, starty = \dofyy, length x = \arrlen,  length y = \arrlen,
  offset x= 0.0cm, offset y= 0.0cm, offset rotation xdir= 0cm, offset rotation ydir= 0cm,
  start angle = \rotdofstartangle, end angle = \rotdofendangle, radius = \arrrad, xstring = $i$, ystring = $i+1$, rstring = $i+2$, font size = \tiny,
  rotation=0, font rotation=0,
  font rotation for x = \dofxrotation, font rotation for y = \dofyrotation, font rotation for r = \dofrrotation,
  arrow size ratio=0.5,
  ]  
}{}

\end{scope}
}

%\framestructure[number of storys=5,
%number of bays=4,
%story height=1.25cm,
%bay width=2cm,
%startX=0cm,
%startY=0cm,
%frame line thickness = 1.2pt,
%support width = 0.5cm,
%support height = 0.2cm,
%support line thickness = 1.5pt,
%show supports = 4,
%isolator width = 0.4cm,
%isolator thickness = 0.2cm,
%isolator line thickness = 1.5pt,
%foundation thickness = 0.6cm]
%\end{tikzpicture}

