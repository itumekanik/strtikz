\pgfkeys{
/springs/.is family, /springs,
default/.style={
startx = 0cm,
starty = 0cm,
start rigid length = 2cm,
end rigid length = 1.5cm,
rigid bar length = 1cm,
rigid end offset ratio = 0.05,
spring length = 2cm,
spring zigzag segment length = 0.2cm,
rotational spring diameter = 0.0005cm,
rotational spring cycle number = 4,
horizontal spring pre length = 0.2cm,
horizontal spring end length = 0.3cm,
vertical spring pre length ratio = 0.25,
vertical spring tick ratio = 0.4},
startx/.store in = \startx,
starty/.store in = \starty,
start rigid length/.store in = \strl,
end rigid length/.store in = \enrl,
rigid bar length/.store in = \rbl,
rigid end offset ratio/.store in = \reor,
spring length/.store in = \splen,
spring zigzag segment length/.store in = \segl,
rotational spring diameter/.store in = \rsdi,
rotational spring cycle number/.store in = \rscn,
horizontal spring pre length/.store in = \hspl,
horizontal spring end length/.store in = \hsel,
vertical spring pre length ratio/.store in = \vsplr,
vertical spring tick ratio/.store in = \vstr,
}
\newcommand{\springs}[1][]{
\pgfkeys{/springs, default, #1}
\tikzmath{
coordinate \c;
real \startx, \starty;
real \strl, \rbl, \enrl, \splen, \segl, \reor;
real \rsdi, \rscn, \rspl, \rsel, \rssl;
real \vsplr, \vstr, \hspl, \hsel;
real \hssl, \vstu, \vspl, \vsel, \vssl, \rssx, \rssxx, \rstrd;
%Conversion all units to the unit pt
\startx = \startx;
\starty = \starty;
\strl=\strl;
\rbl=\rbl;
\enrl=\enrl;
\splen=\splen;
\segl=\segl;
\reor=\reor;
\rsdi=\rsdi;
\rscn=\rscn;
\hspl=\hspl;
\hsel=\hsel;
\vsplr=\vsplr;
\vstr=\vstr;
%End conversion, All values are now in pt units
\hssl = \segl;           %horizontal spring segment length
\vstu = \vstr*\rbl;      %vertical spring tick up
\vspl = \vsplr*\vstu;    %vertical spring pre length
\vsel = \vspl;           %vertical spring end length
\vssl = \segl;           %vertical spring segment length (pt)
\rssx = \strl+\splen/2; 
\rssxx = \rssx + \rscn*2*pi*\rscn*2*pi*\rsdi;
\rstrd = \strl+\splen-\rssxx; %rotational spring to right rigid bar distance
\c1 = (0,0);
\c2 = (\strl pt,0 pt);
\c3 = (\strl pt,\rbl pt);
\c4 = (\strl pt,-\rbl pt);
\c5 = (\strl+\splen pt,0 pt);
\c6 = (\strl+\splen+\enrl pt,0 pt);
\c7 = (\strl+\splen pt,\rbl pt);
\c8 = (\strl+\splen pt,-\rbl pt);
}
\begin{scope}[x=1pt, y=1pt, xshift = \startx, yshift=\starty]; % Draw everything in pt units
\tikzset{hspring/.style={thick,decorate,decoration={zigzag,pre length=\hspl pt,post length=\hsel pt,segment length=\hssl pt}}}
\tikzset{vspring/.style={thick,decorate,decoration={zigzag,pre length=\vspl pt,post length=\vsel pt,segment length=\vssl pt}}}
\draw [line width=4pt] (\c1) -- node[anchor=south] {\small $l_1$}(\c2) ;
\draw [line width=4pt] (\c3) -- (\c4);
\draw [line width=4pt] (\c5) -- node[anchor=south] {\small $l_2$}(\c6);
\draw [line width=4pt] (\c7) -- (\c8);
\draw [hspring] (\c3) ++(0,-\rbl*\reor) -- node[anchor=south] {\small $k_x$}++(\splen,0);
\draw [thick] (\c2) -- ++(\splen/4,0) -- ++(0,\vstu) -- ++(\splen/4,0);
\draw [vspring] (\c2) ++(\splen/4,0) ++(0,\vstu) ++(\splen/4,0) -- node[very near end, left] {\small $k_y$}++(0,-\vstu*2);
\draw [thick] (\c2) ++(\splen/2,-\vstu) -- ++(\splen/4,0) -- ++(0,\vstu) -- ++(\splen/4,0);
\draw [thick,domain=0:\rscn*2*pi,variable=\t,smooth,samples=75,xshift=\rssx,yshift=-\rbl*(1-\reor)] plot ({\t r}: {\rsdi*\t*\t});
\draw [thick](\c4) ++(0,\rbl*\reor) -- node[near end, below] {\small $k_\theta$}++(\splen/2,0);
\draw [thick](\rssxx,-\rbl*(1-\reor) -- ++(\rstrd,0);
\draw [->](\strl+\splen+\enrl+1.0,0) -- +(0.8cm,0) node[near end, above]{\small $x$};
\draw [->](0,1.2 cm) -- +(0,0.8 cm) node[near end, left]{\small $y$};
\end{scope}
}

%\begin{tikzpicture}
%\springs[startx = 2cm,
%starty = 3cm,
%start rigid length = 1cm,
%end rigid length = 1.5cm,
%rigid bar length = 1cm,
%rigid end offset ratio = 0.2,
%spring length = 2cm,
%spring zigzag segment length = 0.2cm,
%rotational spring diameter = 0.0004cm,
%rotational spring cycle number = 4,
%horizontal spring pre length = 0.3cm,
%horizontal spring end length = 0.3cm,
%vertical spring pre length ratio = 0.5,
%vertical spring tick ratio = 0.5]
%\draw (0,0) -- (1cm,1cm);
%\end{tikzpicture}