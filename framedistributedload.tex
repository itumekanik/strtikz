\pgfkeys{
 /framedistributedload/.is family, /framedistributedload,
  default/.style={
  startx = 0cm,
  starty = 0cm,
  endx = 5cm,
  endy = 5cm,
  number of arrows = 10,
  height of arrows = 0.5cm,
  space = 0.1cm,
  start ratio = 0.02,
  end ratio = 0.02,
  load text = $\SI{10}{\kilo\newton/\meter}$},
  startx/.estore in = \startx,
  starty/.estore in = \starty,
  endx/.estore in = \endx,
  endy/.estore in = \endy,
  number of arrows/.estore in = \narr,
  height of arrows/.estore in = \harr,
  space/.estore in = \darr,
  start ratio/.estore in = \sarr,
  end ratio/.estore in = \earr,
  load text/.estore in = \aload,
}

%\startx: starting coordinate x
%\starty: starting coordinate y
%\endx: end coordinate x
%\endy: end coordinate y
%\narr: number of arrows
%\harr: height of arrows
%\darr: distance of arrows from the frame
%\sarr: start(first) arrow start ratio
%\earr: end (last) arrow start ratio
\newcommand{\framedistributedload}[1][]{
\pgfkeys{/framedistributedload, default, #1}
\tikzmath{
real \startx, \starty, \endx, \endy;
real \angl, \unx, \uny, \dxf, \dyf, \dxa, \dya, \aaa, \bbb;
real \x, \y;
real \wx, \wy, \harr, \sarr, \earr, \lenf;
int \narr, \narrmo, \i, \j;
real \rnarr;
%Conversion all units to the unit pt
\startx = \startx;
\starty = \starty;
\endx = \endx;
\endy = \endy;
\harr = \harr;
\darr = \darr;
%End conversion, All values are now in pt units.
 \dxf = (\endx-\startx); %dx for the whole frame
 \dyf = (\endy-\starty); %dy for the whole frame
 \lenf = veclen(\dxf, \dyf);
 \angl = atan2(\dyf,\dxf);
 \dxa = \dxf*(1-(\sarr+\earr)); %dx for the start and end of arrow
 \dya = \dyf*(1-(\sarr+\earr)); %dy for the start and end of arrow
 \lena = veclen(\dxa, \dya);
 \rnarr = \narr;
 \wx = (\dxa)/(\rnarr-1); %dx for the single arrow
 \wy = (\dya)/(\rnarr-1); %dx for the single arrow
 \x{1} = \startx + \dxf*(\sarr);
 \y{1} = \starty + \dyf*(\sarr);
 \x{\narr} = \endx - \dxf*(\earr);
 \y{\narr} = \endy - \dyf*(\earr);
 for \i in {2,...,{\narr-1}}{
 \x{\i} = \x{1} + (\i-1)*\wx;
 \y{\i} = \y{1} + (\i-1)*\wy;
 };
 \narrmo = \narr-1;
 \loc = ifthenelse(\harr>0,"above","below");
% print {$\dxf$};
}
\begin{scope}[x=1pt, y=1pt]; % Drawing everything in pt units
\foreach \i in {2,...,\narrmo}{
\draw [<-, line width=0.5] (\x{\i}, \y{\i}) ++(90+\angl:\darr) -- ++(90+\angl:\harr);}
\draw [<->, line width=0.5] (\x{1},\y{1}) ++(90+\angl:\darr)--
++(90+\angl:\harr) -- node[\loc,midway,sloped]{\aload}
++(\angl:\lena)-- ++(-90+\angl:\harr);
\end{scope}
}

%\begin{tikzpicture}
%\draw [line width=1](0 cm, 0 cm) -- (5 cm, 5 cm);
%\framedistributedload[startx = 0cm, starty = 0cm,
%endx = 5cm, endy = 5cm,
%number of arrows = 10,
%height of arrows = 0.5cm,
%space = 0.1cm,
%start ratio = 0.02,
%end ratio = 0.02,
%load text = $\SI{10}{\kilo\newton/\meter}$]
%\end{tikzpicture}

