\pgfkeys{
/rotspring/.is family, /rotspring,
default/.style={
startx = 0cm,
starty = 0cm,
rotation = 0,
start rigid length = 2cm,
end rigid length = 1.5cm,
rotational spring diameter = 0.0005cm,
rotational spring cycle number = 4,
text = $k_\theta$},
startx/.store in = \startx,
starty/.store in = \starty,
rotation/.store in = \rotn,
start rigid length/.store in = \strl,
end rigid length/.store in = \enrl,
rotational spring diameter/.store in = \rsdi,
rotational spring cycle number/.store in = \rscn,
text/.store in = \sptext}
\newcommand{\rotspring}[1][]{
\pgfkeys{/rotspring, default, #1}
\tikzmath{
real \startx, \starty;
real \strl, \enrl, \rotn;
real \rsdi, \rscn;
real \rssx, \rssxx, \rstrd;
real \rsed;
%Conversion all units to the unit pt
\startx = \startx;
\starty = \starty;
\strl=\strl;
\enrl=\enrl;
\rsdi=\rsdi;
\rscn=\rscn;
%End conversion, All values are now in pt units
\rsed = \rscn*2*pi*\rscn*2*pi*\rsdi;
}
\begin{scope}[x=1pt, y=1pt, xshift = \startx, yshift=\starty, rotate=\rotn]; % Draw everything in pt units
\draw [thick](0,0) -- node[near end, below] {\small \sptext}++(\strl,0);
\draw [thick,domain=0:\rscn*2*pi,variable=\t,smooth,samples=75,xshift=\strl] plot ({\t r}: {\rsdi*\t*\t}) -- ++(\enrl-\rsed,0);;
\end{scope}
}

%\begin{tikzpicture}
%\rotspring[startx = 2cm,
%starty = 3cm,
%rotation = 30,
%start rigid length = 0.5cm,
%end rigid length = 0.5cm,
%rotational spring diameter = 0.0004cm,
%rotational spring cycle number = 3,
%text = $k_\theta$]
%\draw (0,0) -- (1cm,1cm);
%\end{tikzpicture}
