\pgfkeys{
/rotspring/.is family, /rotspring,
default/.style={
  startx = 0cm,
  starty = 0cm,
  rotation = 0,
  start rigid length = 2cm,
  end rigid length = 1.5cm,
  rotational spring diameter = 0.0005cm,
  rotational spring cycle number = 4,
  text = $k_\theta$,
  text location = above,
  text location offset = 6pt,
  rigid thickness = 2pt,
  spring thickness = 1.5pt,
  spring direction mode = 1,
  radius coefficient = 1,
  rigid left text = {$L_1=0$},
  rigid right text = {$L_2=0$},
  rigid text offset = -2pt},
  startx/.store in = \startrotx,
  starty/.store in = \startroty,
  rotation/.store in = \rotn,
  start rigid length/.store in = \strl,
  end rigid length/.store in = \enrl,
  rotational spring diameter/.store in = \rsdi,
  rotational spring cycle number/.store in = \rscn,
  text/.store in = \sptext,
  text location/.store in = \textloc,
  text location offset/.store in = \textlocoff,
  rigid thickness/.store in = \riglinet,
  spring thickness/.store in = \sprlinet,
  spring direction mode/.store in = \sprdirmode,
  radius coefficient/.store in = \radcoeff,
  rigid left text/.store in = \rigidlefttext,
  rigid right text/.store in = \rigidrighttext,
  rigid text offset/.store in = \riglentextoff,
}
\newcommand{\rotspring}[1][]{
\pgfkeys{/rotspring, default, #1}
\tikzmath{
real \startrotx, \startroty;
real \strl, \enrl, \rotn;
real \rsdi, \rscn;
real \rssx, \rssxx, \rstrd;
real \rsed;
real \riglinet, \sprlinet, \radcoeff;
%Conversion all units to the unit pt
\startrotx = \startrotx;
\startroty = \startroty;
\strl=\strl;
\enrl=\enrl;
\rsdi=\rsdi;
\rscn=\rscn;
\textlocoff = \textlocoff;
\riglinet = \riglinet;
\sprlinet = \sprlinet;
\riglentextoff = \riglentextoff;
%End conversion, All values are now in pt units
\rsed = \radcoeff*\rscn*2*pi*\rscn*2*pi*\rsdi;
\springdomain = \rscn*2*pi;
\samplepercycle = 50;
}
\begin{scope}[x=1pt, y=1pt, xshift = \startrotx, yshift=\startroty, rotate=\rotn]; % Draw everything in pt units
\ifthenelse{\sprdirmode=1}{
\draw [line width=\riglinet](0,0) -- ++(\strl,0) node[midway, rotate=\rotn, above=\riglentextoff] {\rigidlefttext};
\draw [line width=\sprlinet,
domain=0:\springdomain,variable=\t,smooth,samples=\rscn*\samplepercycle,xshift=\strl]
plot ({\t r}: {\rsdi*\t*\t*\radcoeff});
\draw (\strl,0) node[above, \textloc=\textlocoff] {\sptext};
\draw [line width=\riglinet](\strl+\rsed-\sprlinet/2,0)--(\enrl+\strl,0)
node[midway,  rotate=\rotn, above=\riglentextoff] {\rigidrighttext};
}{};

\ifthenelse{\sprdirmode=2}{
\draw [line width=\riglinet](0,0) -- ++(\strl,0);
\draw [line width=\sprlinet,
domain=0:-\springdomain,variable=\t,smooth,samples=\rscn*\samplepercycle,xshift=\strl]
plot ({\t r}: {\rsdi*\t*\t*\radcoeff});
\draw [line width=\riglinet](\strl+\rsed-\sprlinet/2,0)-- node[at start, \textloc=\textlocoff] {\sptext} (\enrl+\strl,0);
}{};

\ifthenelse{\sprdirmode=3}{
\draw [line width=\riglinet](0,0) -- ++(\strl-\rsed+\sprlinet/2,0);
\draw [line width=\sprlinet,
domain=0:\springdomain,variable=\t,smooth,samples=\rscn*\samplepercycle,xshift=\strl]
plot ({\t r}: {-\rsdi*\t*\t*\radcoeff});
\draw [line width=\riglinet](\strl,0)-- node[at start, \textloc=\textlocoff] {\sptext} (\enrl+\strl,0);
}{};

\ifthenelse{\sprdirmode=4}{
\draw [line width=\riglinet](0,0) -- ++(\strl-\rsed+\sprlinet/2,0);
\draw [line width=\sprlinet,
domain=0:-\springdomain,variable=\t,smooth,samples=\rscn*\samplepercycle,xshift=\strl]
plot ({\t r}: {-\rsdi*\t*\t*\radcoeff});
\draw [line width=\riglinet](\strl,0)-- node[at start, \textloc=\textlocoff] {\sptext} (\enrl+\strl,0);
}{};
\end{scope}
}

%\begin{tikzpicture}
%\rotspring[startx = 2cm,
%starty = 3cm,
%rotation = 30,
%start rigid length = 0.5cm,
%end rigid length = 0.5cm,
%rotational spring diameter = 0.0004cm,
%rotational spring cycle number = 3,
%text = $k_\theta$]
%\draw (0,0) -- (1cm,1cm);
%\end{tikzpicture}
