\pgfkeys{
 /dofs/.is family, /dofs,
  default/.style={
  startx = 0cm,
  starty = 0cm,
  length x = 1cm,
  length y = 1cm,
  offset = 0.1cm,
  start angle = 110,
  end angle = 340,  
  radius = 0.5cm,
  xstring = $x$,
  ystring = $y$,  
  rstring = $z$,
  font size = \normalsize,
  x location = below,
  y location = right,  
  r location = right,
  x location offset = 0pt,
  y location offset = 0pt,
  r location offset = 0pt,
  rotation = 0},
  startx/.store in = \xx,
  starty/.store in = \yy,
  length x/.store in = \lx,
  length y/.store in = \ly,
  offset/.store in = \tdof,
  start angle/.store in = \sa,
  end angle/.store in = \ea,
  radius/.store in = \radi,
  xstring/.store in = \dofx,
  ystring/.store in = \dofy,
  rstring/.store in = \dofr,
  font size/.store in = \fsize,
  x location/.store in = \xloc,
  y location/.store in = \yloc,
  r location/.store in = \rloc,
  x location offset/.store in = \xlocoff,
  y location offset/.store in = \ylocoff,
  r location offset/.store in = \rlocoff,
  rotation/.store in = \rot}

\newcommand{\dofs}[1][]{
\pgfkeys{/dofs, default, #1}

\tikzmath{
real \xx, \yy, \lx, \ly;
real \tdof, \radi;
coordinate \cjoint; % coordinate of the joint
%Conversion all units to the unit pt
\xx = \xx;
print {\xx};
\yy = \yy;
\lx = \lx;
\ly = \ly;
\tdof = \tdof;
\radi = \radi;
\xlocoff = \xlocoff;
\ylocoff = \ylocoff;
\rlocoff = \rlocoff;
%End conversion, All values are now in pt units.
%print {$\xx -$};
%print {$\yy -$};
}
\begin{scope}[x=1pt, y=1pt, xshift=\xx, yshift=\yy, rotate=\rot]; % Drawing everything in pt units
\draw [arrows={->[scale=0.5]}] (\tdof,0) -- node[at end, \xloc=\xlocoff]{\fsize \dofx} ++(\lx,0);
\draw [arrows={->[scale=0.5]}] (0, \tdof) -- node[at end, \yloc=\ylocoff]{\fsize \dofy}++(0,\ly);
\draw [arrows={->[scale=0.5]}] (\sa:\radi) ++(0,0) arc [start angle=\sa, end angle=\ea, radius=\radi]
node[very near end, \rloc=\rlocoff]{\fsize \dofr};
\end{scope}
}

%\begin{tikzpicture}
%\dofs[startx = 0cm,
%  starty = 0cm,
%  dx = 1in,
%  dy = 1cm,
%  offset = 0.0cm,
%  start angle = 110,
%  end angle = 340,  
%  radius = 1cm,
%  xstring = $x$,
%  ystring = $y$,  
%  rstring = $\theta$,
%  font size = \scriptsize,
%  x location = left,
%  y location = right,  
%  r location = right,
%  rotation=45]
%\end{tikzpicture}