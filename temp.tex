\documentclass{ltxdoc}
\usepackage[T1]{fontenc}
\usepackage{tabularx}
\usepackage{syntax}
\usepackage{varioref}
\usepackage{color}
\usepackage{booktabs}
\usepackage{alltt}
\usepackage{textcomp}
\usepackage[bookmarksopen=true]{hyperref}

% Define this document's metadata.
\def\ncfileversion{2.0}
\def\ncfiledate{2010/06/01}

\title{The \textsf{newcommand.py} utility\thanks{\textsf{newcommand.py}
         has version number \ncfileversion, last revised \ncfiledate.}}
\author{\href{mailto:scott+nc@pakin.org}{Scott Pakin} \\
        \href{mailto:scott+nc@pakin.org}{scott+nc@pakin.org}}
\date{\ncfiledate}
\hypersetup{%
  pdftitle={The newcommand.py utility},
  pdfauthor={Scott Pakin <scott+nc@pakin.org>},
  pdfsubject={Creating user-defined macros with more flexible argument
    processing},
  pdfkeywords={LaTeX macros, optional arguments, newcommand, parenthesized
    arguments, starred commands, Python}
}

% Help prevent weird line breaks in URLs
\def\UrlBreaks{}
\def\UrlBigBreaks{\do/}

% Define some useful shortcuts.
\newcommand*{\ncpy}{\texttt{newcommand.py}}      % The name of the script
\newcommand*{\usercmd}[1]{\textcolor{blue}{#1}}  % User command entry
\newcommand{\prototype}[1]{{%                    % Prompt and user entry
  \bigskip
  \noindent
  \ttfamily\small\raggedright
  \hangafter=1\hangindent=4em
  ~~~~\% Prototype:~\textcolor{blue}{#1}\strut\par
  \vspace*{-0.5\baselineskip}%
}}

%%%%%%%%%%%%%%%%%%%%%%%%%%%%%%%%%%%%%%%%%%%%%%%%%%%%%%%%%%%%%%%%%%%%%%%%%%%

\begin{document}
\maketitle
\sloppy

\begin{abstract}
  \LaTeX's \cs{newcommand} is fairly limited in the way it processes
  optional arguments, but the \TeX\ alternative, a batch of \cs{def}s
  and \cs{futurelet}s, can be overwhelming to the casual \LaTeX\ user.
  \ncpy\ is a Python program that automatically generates
  \LaTeX\ macro definitions for macros that require more powerful
  argument processing than \cs{newcommand} can handle.  \ncpy\ is
  intended for \LaTeX\ advanced beginners (i.e., those who know how to
  use \cs{newcommand} but not internal \LaTeXe\ macros like
  |\@ifnextchar|) and for more advanced users who want to save some
  typing when defining complex macros.
\end{abstract}

\section{Introduction}

\LaTeX's \cs{newcommand} is a rather limited way to define new macros.
Only one argument can be designated as optional, it must be the first
argument, and it must appear within square brackets.  Defining macros
that take multiple optional arguments or in which an optional argument
appears in the middle of the argument list is possible but well beyond
the capabilities of the casual \LaTeX\ user.  It requires using
\TeX\ primitives such as \cs{def} and \cs{futurelet} and/or
\LaTeXe\ internal macros such as \cs{@ifnextchar}.

\ncpy\ is a Python program that reads a specification of an argument
list and automatically produces \LaTeX\ code that processes the
arguments appropriately.  \ncpy\ makes it easy to define
\LaTeX\ macros with more complex parameter parsing than is possible
with \cs{newcommand} alone.  Note that you do need to have Python
installed on your system to run \ncpy.  Python is freely available for
download from \url{http://www.python.org/}.

To define a \LaTeX\ macro, one gives \ncpy\ a macro description
written in a simple specification language.  The description
essentially lists the required and optional arguments and, for each
optional argument, the default value.  The next section of this
document describes the syntax and provides some examples, but for now,
let's look at how one would define the most trivial macro possible,
one that takes no arguments.  Enter the following at your operating
system's prompt:

\begin{alltt}
    \usercmd{newcommand.py "MACRO trivial"}
\end{alltt}

\noindent
(Depending on your system, you may need to prefix that command with
``|python|''.)  The program should output the following \LaTeX\ code
in response:

\begin{verbatim}
    % Prototype: MACRO trivial
    \newcommand{\trivial}{%
      % Put your code here.
    }
\end{verbatim}

\noindent
Alternatively, you can run \ncpy\ interactively, entering macro
descriptions at the ``|% Prototype:|'' prompt:

\prototype{MACRO trivial}
\begin{verbatim}
    \newcommand{\trivial}{%
      % Put your code here.
    }
    % Prototype:
\end{verbatim}

\noindent
Enter your operating system's end-of-file character (Ctrl-D in Unix or
Ctrl-Z in Windows) to exit the program.

While you certainly don't need \ncpy\ to write macros that are as
trivial as \cs{trivial}, the previous discussion shows how to run the
program and the sort of output that you should expect.  There will
always be a ``\texttt{Put your code here}'' comment indicating where
you should fill in the actual macro code.  At that location, all of
the macro's parameters---both optional and required---will be defined
and can be referred to in the ordinary way: |#1|, |#2|, |#3|, etc.


\section{Usage}

As we saw in the previous section, macros are defined by the word
``|MACRO|'' followed by the macro name, with no preceding backslash.
In this section we examine how to specify increasingly sophisticated
argument processing using \ncpy.


\subsection{Required arguments}

Required arguments are entered as~|#1|, |#2|, |#3|, \dots, with no
surrounding braces:

\prototype{MACRO required \#1 \#2 \#3 \#4 \#5}
\begin{verbatim}
    \newcommand{\required}[5]{%
      % Put your code here.
      % You can refer to the arguments as #1 through #5.
    }
\end{verbatim}

Parameters must be numbered in monotonically increasing order,
starting with~|#1|.  Incorrectly ordered parameters will produce an
error message:

\prototype{MACRO required \#1 \#3 \#4}
\begin{verbatim}
                                   ^
    newcommand.py: Expected parameter 2 but saw parameter 3.
\end{verbatim}


\subsection{Optional arguments}

Optional arguments\label{par:optional-args} are written as either
``|OPT[|\meta{param}|=|\linebreak[1]|{|\meta{default}|}]|'' or
``|OPT(|\meta{param}|=|\linebreak[1]|{|\meta{default}|})|''.  In the
former case, square brackets are used to offset the optional argument;
in the latter case, parentheses are used.  \meta{param} is the
parameter number (|#1|, |#2|, |#3|, \dots), and \meta{default} is the
default value for that parameter.  Note that curly braces are required
around \meta{default}.

\prototype{MACRO optional OPT[\#1=\{maybe\}]}
\begin{verbatim}
    \newcommand{\optional}[1][maybe]{%
      % Put your code here.
      % You can refer to the argument as #1.
    }
\end{verbatim}

Up to this point, the examples have been so simple that \ncpy\ is
overkill for entering them.  We can now begin specifying constructs
that \LaTeX's \cs{newcommand} can't handle, such as a parenthesized
optional argument, an optional argument that doesn't appear at the
beginning of the argument list, and multiple optional arguments:

\prototype{MACRO parenthesized OPT(\#1=\{abc\})}
\begin{verbatim}
    \makeatletter
    \newcommand{\parenthesized}{%
      \@ifnextchar({\parenthesized@i}{\parenthesized@i({abc})}%
    }

    \def\parenthesized@i(#1){%
      % Put your code here.
      % You can refer to the argument as #1.
    }
    \makeatother
\end{verbatim}


\prototype{MACRO nonbeginning \#1 OPT[\#2=\{abc\}]}
\begin{verbatim}
    \makeatletter
    \newcommand{\nonbeginning}[1]{%
      \@ifnextchar[{\nonbeginning@i{#1}}{\nonbeginning@i{#1}[{abc}]}%
    }

    \def\nonbeginning@i#1[#2]{%
      % Put your code here.
      % You can refer to the arguments as #1 and #2.
    }
    \makeatother
\end{verbatim}

\prototype{MACRO multiple OPT[\#1=\{abc\}] OPT[\#2=\{def\}]}
\begin{verbatim}
    \makeatletter
    \newcommand{\multiple}[1][abc]{%
      \@ifnextchar[{\multiple@i[{#1}]}{\multiple@i[{#1}][{def}]}%
    }

    \def\multiple@i[#1][#2]{%
      % Put your code here.
      % You can refer to the arguments as #1 and #2.
    }
    \makeatother
\end{verbatim}

The template for optional arguments that was
shown~\vpageref[above]{par:optional-args} stated that optional
arguments contain a ``\meta{param}|={|\meta{default}|}|''
specification.  In fact, optional arguments can contain
\emph{multiple} ``\meta{param}|={|\meta{default}|}|'' specifications,
as long as they are separated by literal text:

\prototype{MACRO multiopt OPT(\#1=\{0\},\#2=\{0\})}
\begin{verbatim}
    \makeatletter
    \newcommand{\multiopt}{%
      \@ifnextchar({\multiopt@i}{\multiopt@i({0},{0})}%
    }

    \def\multiopt@i(#1,#2){%
      % Put your code here.
      % You can refer to the arguments as #1 and #2.
    }
    \makeatother
\end{verbatim}

\noindent
In that example, \cs{multiopt} takes an optional parenthesized
argument.  If omitted, it defaults to |(0,0)|.  If provided, the
argument must be of the form ``|(|\meta{x}|,|\meta{y}|)|''.  In either
case, the comma-separated values within the parentheses are parsed
into~|#1| and~|#2|.  Contrast that with the following:

\prototype{MACRO multiopt OPT(\#1=\{0,0\})}
\begin{verbatim}
    \makeatletter
    \newcommand{\multiopt}{%
      \@ifnextchar({\multiopt@i}{\multiopt@i({0,0})}%
    }

    \def\multiopt@i(#1){%
      % Put your code here.
      % You can refer to the argument as #1.
    }
    \makeatother
\end{verbatim}

The optional argument still defaults to |(0,0)|, but |#1| receives
\emph{all} of the text that lies between the parentheses;
\cs{multiopt} does not parse it into two comma-separated values
in~|#1| and~|#2|, as it did in the previous example.

\bigskip

The \meta{default} text in an |OPT| term can reference any macro
parameter introduced before the |OPT|\@.  Hence, the following defines
a macro that accepts a required argument followed by an optional
argument.  The default value of the optional argument is the value
provided for the required argument:

\prototype{MACRO paramdefault \#1 OPT[\#2=\{\#1\}]}
\begin{verbatim}
    \makeatletter
    \newcommand{\paramdefault}[1]{%
      \@ifnextchar[{\paramdefault@i{#1}}{\paramdefault@i{#1}[{#1}]}%
    }

    \def\paramdefault@i#1[#2]{%
      % Put your code here.
      % You can refer to the arguments as #1 and #2.
    }
    \makeatother
\end{verbatim}


\subsection{Literal text}

In addition to required and optional parameters, it is also possible
to specify text that must appear literally in the macro call.  Merely
specify it within curly braces:

\prototype{MACRO textual \#1 \{ and \} \#2 \{.\}}
\begin{verbatim}
    \makeatletter
    \newcommand{\textual}[1]{%
      \textual@i{#1}%
    }

    \def\textual@i#1 and #2.{%
      % Put your code here.
      % You can refer to the arguments as #1 and #2.
    }
    \makeatother
\end{verbatim}

\noindent
A macro such as \cs{textual} can be called like this:

\begin{verbatim}
    \textual {Milk} and {cookies}.
\end{verbatim}

\noindent
Actually, in that example, because both |Milk| and |cookies| are
delimited on the right by literal text, \TeX\ can figure out how to
split \cs{textual}'s argument into~|#1| and~|#2| even if the curly
braces are omitted:

\begin{verbatim}
    \textual Milk and cookies.
\end{verbatim}

\subsection{Starred macros}

The names of some \LaTeX\ macros can be followed by an optional
``|*|'' to indicate a variation on the normal processing.  For
example, \cs{vspace}, which introduces a given amount of vertical
space, discards the space if it appears at the top of the page.
|\vspace*|, in contrast, typesets the space no matter where it
appears.  \ncpy\ makes it easy for users to define their own starred
commands:

\prototype{MACRO starred * \#1 \#2}
\begin{verbatim}
    \makeatletter
    \newif\ifstarred@star
    \newcommand{\starred}{%
      \@ifstar{\starred@startrue\starred@i*}{\starred@starfalse\starred@i*}%
    }

    \def\starred@i*#1#2{%
      \ifstarred@star
        % Put code for the "*" case here.
      \else
        % Put code for the non-"*" case here.
      \fi
      % Put code common to both cases here (and/or above the \ifstarred@star).
      % You can refer to the arguments as #1 and #2.
    }
    \makeatother
\end{verbatim}

\noindent
Note that unlike the generated code shown up to this point, the code
for starred macros includes \emph{multiple} placeholders for user
code.

The ``|*|'' in a starred macro does not have to immediately follow the
macro name; it can appear anywhere in the macro specification.
However, \ncpy\ currently limits macros to at most one asterisk.

Embedding an asterisk within curly braces causes it to be treated not
as an optional character but as (required) literal text.  Contrast the
preceding example with the following one:

\prototype{MACRO starred \{*\} \#1 \#2}
\begin{verbatim}
    \makeatletter
    \newcommand{\starred}{%
      \starred@i%
    }

    \def\starred@i*#1#2{%
      % Put your code here.
      % You can refer to the arguments as #1 and #2.
    }
    \makeatother
\end{verbatim}

The asterisk in that definition of \cs{starred} must be included in
every macro invocation or \TeX\ will abort with a ``\texttt{Use of
  \string\starred@i doesn't match its definition}'' error.


\subsection{More than nine arguments}

\TeX\ imposes a limit of nine arguments per macro.  Internally,
``|#|'' is expected to be followed by exactly one digit, which means
that ``|#10|'' refers to argument~|#1| followed by the character~|0|.
Fortunately, it's rare that a macro needs more than nine arguments and
rarer still that those arguments are not better specified as a list of
\meta{key}|=|\meta{value} pairs, as supported by the \textsf{keyval}
package and many other \LaTeX\ packages.

If large numbers of arguments are in fact necessary, \ncpy\ does let
you specify them.  The trick that the generated code uses is to split
the macro into multiple macros, each of which takes nine or fewer
arguments and stores the value of each argument in a variable that can
later be accessed.  Because digits are awkward to use in macro names,
\ncpy\ uses roman numerals to name arguments in the case of more than
nine arguments: |\|\meta{name}|@arg@i| for |#1|,
|\|\meta{name}|@arg@ii| for |#2|, |\|\meta{name}|@arg@iii| for |#3|,
|\|\meta{name}|@arg@iv| for |#4|, and so forth.  The following example
takes 14 required arguments and one optional argument (which defaults
to the string ``|etc|''):

\prototype{MACRO manyargs \#1 \#2 \#3 \#4 \#5 \#6 \#7 \#8 \#9 \#10
  \#11 \#12 \#13 \#14 OPT[\#15=\{etc\}]}
\begin{verbatim}
    \makeatletter
    \newcommand{\manyargs}[9]{%
      \def\manyargs@arg@i{#1}%
      \def\manyargs@arg@ii{#2}%
      \def\manyargs@arg@iii{#3}%
      \def\manyargs@arg@iv{#4}%
      \def\manyargs@arg@v{#5}%
      \def\manyargs@arg@vi{#6}%
      \def\manyargs@arg@vii{#7}%
      \def\manyargs@arg@viii{#8}%
      \def\manyargs@arg@ix{#9}%
      \manyargs@i
    }

    \def\manyargs@i#1#2#3#4#5{%
      \def\manyargs@arg@x{#1}%
      \def\manyargs@arg@xi{#2}%
      \def\manyargs@arg@xii{#3}%
      \def\manyargs@arg@xiii{#4}%
      \def\manyargs@arg@xiv{#5}%
      \@ifnextchar[{\manyargs@ii}{\manyargs@ii[{etc}]}%
    }

    \def\manyargs@ii[#1]{%
      \def\manyargs@arg@xv{#1}%
      % Put your code here.
      % You can refer to the arguments as \manyargs@arg@i through \manyargs@arg@xv.
    }
    \makeatother
\end{verbatim}

The current version of \ncpy\ is limited to 4000 arguments, which
should be more than enough for most purposes.


\subsection{Summary}

A macro is defined in \ncpy\ with:

\begin{center}
  \texttt{MACRO} \meta{name} \meta{arguments}
\end{center}

\noindent
in which \meta{name} is the name of the macro, and \meta{arguments} is
zero or more of the following:

\begin{center}
\renewcommand{\arraystretch}{1.1}
\begin{tabularx}{\linewidth}{@{}lXl@{}}
  \toprule
  \multicolumn{1}{@{}c}{Argument} &
  \multicolumn{1}{c}{Meaning} &
  \multicolumn{1}{c@{}}{Example} \\
  \midrule

  |#|\meta{number} & Parameter (required) & |#1| \\
  \marg{text}      & Literal text (required)     & |{+}| \\
  |OPT[#|\meta{number}|=|\marg{text}|]| & Parameter (optional, with default) &
    |OPT[#1={tbp}]| \\
  |OPT(#|\meta{number}|=|\marg{text}|)| & Same as the above, but with
    parentheses instead of brackets & |OPT(#1={tbp})| \\
  |*|              & Literal asterisk (optional) & |*| \\
  \bottomrule
\end{tabularx}
\end{center}

Within an |OPT| argument, |#|\meta{number}|=|\marg{text} can be
repeated any number of times, as long as the various instances are
separated by literal text.


\section{Further examples}

\subsection{Mimicking \LaTeX's \texttt{picture} environment}

The \LaTeX\ |picture| environment takes two, parenthesized,
coordinate-pair arguments, the second pair being optional.  Here's how
to define a macro that takes the same arguments as the |picture|
environment and parses them into $x_1$, $y_1$, $x_2$, and~$y_2$
(i.e.,~|#1|--|#4|):

\prototype{MACRO picturemacro \{(\}\#1\{,\}\#2\{)\} OPT(\#3=\{0\},\#4=\{0\})}
\begin{verbatim}
    \makeatletter
    \newcommand{\picturemacro}{%
      \picturemacro@i%
    }

    \def\picturemacro@i(#1,#2){%
      \@ifnextchar({\picturemacro@ii({#1},{#2})}{\picturemacro@ii({#1},{#2})({0},{0})}%
    }

    \def\picturemacro@ii(#1,#2)(#3,#4){%
      % Put your code here.
      % You can refer to the arguments as #1 through #4.
    }
    \makeatother
\end{verbatim}

The first pair of parentheses and the comma are quoted because they
represent required, literal text.


\subsection{Mimicking \LaTeX's \texttt{\string\parbox} macro}

\LaTeX's \cs{parbox} macro takes three optional arguments and two
required arguments.  Furthermore, the third argument defaults to
whatever value was specified for the first argument.  This is easy to
express in \LaTeX\ with the help of \ncpy:

\prototype{MACRO parboxlike OPT[\#1=\{s\}] OPT[\#2=\{\string\relax\}]
  OPT[\#3=\{\#1\}] \#4 \#5}
\begin{verbatim}
    \makeatletter
    \newcommand{\parboxlike}[1][s]{%
      \@ifnextchar[{\parboxlike@i[{#1}]}{\parboxlike@i[{#1}][{\relax}]}%
    }

    \def\parboxlike@i[#1][#2]{%
      \@ifnextchar[{\parboxlike@ii[{#1}][{#2}]}{\parboxlike@ii[{#1}][{#2}][{#1}]}%
    }

    \def\parboxlike@ii[#1][#2][#3]#4#5{%
      % Put your code here.
      % You can refer to the arguments as #1 through #5.
    }
    \makeatother
\end{verbatim}


\subsection{Dynamically changing argument formats}
\label{sec:dynamic-args}

With a little cleverness, it is possible for a macro to accept one of
two completely different sets of arguments based on the values
provided for earlier arguments.  For example, suppose we want to
define a macro, \cs{differentargs} that can be called as either

\begin{verbatim}
    \differentargs*[optarg]{reqarg}
\end{verbatim}

\noindent
or

\begin{verbatim}
    \differentargs{reqarg}(optarg)
\end{verbatim}

\noindent
That is, the presence of an asterisk determines whether
\cs{differentargs} should expect an optional argument in square
brackets followed by a required argument or to expect a required
argument followed by an optional argument in parentheses.

The trick is to create two helper macros: one for the ``|*|'' case
(\cs{withstar}) and the other for the non-``|*|'' case
(\cs{withoutstar}).  \cs{differentargs} can then invoke one of
\cs{withstar} or \cs{withoutstar} based on whether or not it sees an
asterisk.  The following shows how to use \ncpy\ to define
\cs{differentargs}, \cs{withstar}, and \cs{withoutstar} and how to
edit \cs{differentargs} to invoke its helper macros:

\prototype{MACRO differentargs *}
\begin{verbatim}
    \makeatletter
    \newif\ifdifferentargs@star
    \newcommand{\differentargs}{%
      \@ifstar{\differentargs@startrue\differentargs@i*}
              {\differentargs@starfalse\differentargs@i*}%
    }
\end{verbatim}
\begingroup
\small
\begin{alltt}
    \cs{def}\cs{differentargs@i}*\{%
      \cs{ifdifferentargs@star}
        % Put code for the "*" case here.
        \colorbox{yellow}{\cs{let}\cs{next}=\cs{withstar}}
      \cs{else}
        % Put code for the non-"*" case here.
        \colorbox{yellow}{\cs{let}\cs{next}=\cs{withoutstar}}
      \cs{fi}
      % Put code common to both cases here (and/or above the \cs{ifdifferentargs@star}).
      \colorbox{yellow}{\cs{next}}
    \}
    \cs{makeatother}
\end{alltt}
\endgroup

\prototype{MACRO withstar OPT[\#1=\{starry\}] \#2}
\begin{verbatim}
    \newcommand{\withstar}[2][starry]{%
      % Put your code here.
      % You can refer to the arguments as #1 and #2.
    }
\end{verbatim}

\prototype{MACRO withoutstar \#1 OPT(\#2=\{dark\})}
\begin{verbatim}
    \makeatletter
    \newcommand{\withoutstar}[1]{%
      \@ifnextchar({\withoutstar@i{#1}}{\withoutstar@i{#1}({dark})}%
    }

    \def\withoutstar@i#1(#2){%
      % Put your code here.
      % You can refer to the arguments as #1 and #2.
    }
    \makeatother
\end{verbatim}

Note that we edited \cs{differentargs@i} to let \cs{next} be
equivalent to either \cs{withstar} or \cs{withoutstar} based on
whether an asterisk was encountered.  \cs{next} is evaluated outside
of the \cs{ifdifferentargs@star}\dots\linebreak[0]\cs{fi} control
structure.  This rigmarole is necessary because directly calling
\cs{withstar} or \cs{withoutstar} would cause those macros to see
\cs{ifdifferentargs@star}'s \cs{else} or \cs{fi} as their first
argument when they ought to see the text following the
\cs{differentargs} call.


\section{Grammar}

The following is the formal specification of \ncpy's grammar, written
in a more-or-less top-down manner.  Literal values, shown in a
typewriter font, are case-sensitive.  \meta{letter} refers to a letter
of the (English) alphabet.  \meta{digit} refers to a digit.

\setlength{\grammarindent}{7em}
\begin{grammar}
<decl> ::= \[[ "MACRO" <ident> <arglist> \]]

<ident> ::= \[[ \begin{rep} <letter> \end{rep} \]]

<arglist> ::= \[[ \begin{rep} \\ <arg> \end{rep} \]]

<arg> ::= \[[
              \begin{stack}
                <formal> \\
                <quoted> \\
                <optarg> \\
                "*"
              \end{stack}
          \]]

<formal> ::= \[[ "#" \begin{rep} <digit> \end{rep} \]]

<quoted> ::= \[[ "{" <rawtext> "}" \]]

<rawtext> ::= \[[ \begin{rep} \tok{anything except a "{", "}", or "\#"} \end{rep} \]]

<optarg> ::= \[[ "OPT" <delim> <defvals> <delim> \]]

<delim> ::= \[[ \begin{stack} "[" \\ "]" \\ "(" \\ ")" \end{stack} \]]

<defvals> ::= \[[
                  \begin{rep}
                    <defval> \\
                    \begin{stack}
                      <quoted> \\
                      <rawtext>
                    \end{stack}
                  \end{rep}
              \]]

<defval> ::= \[[ <formal> "=" <quoted> \]]
\end{grammar}


\section{Acknowledgements}

I'd like to say thank you to the following people:

\begin{itemize}
  \item John Aycock for writing the
    \href{http://pages.cpsc.ucalgary.ca/~aycock/spark/}{Scanning,
      Parsing, and Rewriting Kit (SPARK)}---the lexer and parser
    underlying \ncpy---and making it freely available and
    redistributable.

  \item Hendri Adriaens for pointing out a bug in the code generated
    by \ncpy.  Previously, bracketed text within a mandatory argument
    could be mistaken for an optional argument.

  \item Tom Potts for reporting a spurious error message caused by the
    processing of |OPT|.  This bug has now been fixed.  Tom Potts also
    proposed the example used in Section~\ref{sec:dynamic-args} in
    which the starred and unstarred versions of a macro take different
    arguments.
\end{itemize}


\section{Copyright and license}

Copyright~\copyright{} 2010, Scott Pakin

\bigskip

This package may be distributed and/or modified under the conditions
of the \LaTeX{} Project Public License, either version~1.3c of this
license or (at your option) any later version.  The latest version of
this license is in:

\begin{center}
  \url{http://www.latex-project.org/lppl.txt}
\end{center}

\noindent
and version~1.3c or later is part of all distributions of \LaTeX{}
version 2006/05/20 or later.

\end{document}