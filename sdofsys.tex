\pgfkeys{
 /sdofsys/.is family, /sdofsys,
  default/.style={
  vertical support line thickness = 1pt,
  horizontal support line thickness = 1pt,  
  spring and damper line thickness = 1pt,
  mass line thickness = 1pt,
  wheel line thickness = 1pt,
  startx = 0cm,
  starty = 0cm,
  vertical support shade thickness = 0.3cm,
  horizontal support shade thickness = 0.3cm,    
  support width = 4cm,
  support height = 1.5cm,
  mass height = 2cm,
  mass width = 2cm,
  massx = 1.5cm,  
  wheel radius = 0.15cm,
  spring location ratio = 0.25;
  spring ratio = 0.15,
  damper width ratio = 0.15,
  damper height ratio = 0.15,
  damper piston size ratio = 0.8,
  damper piston position ratio = 0.4,
  wheel location ratio = 0.25,
  wheel size ratio = 0.8,
  mass text = $m$,
  spring text = $k$,
  damper text = $c$,
  disp line location ratio = 1,
  disp line = 0.8cm,
  disp arrow = 1cm,
  disp text = $x(t)$,
  disp line thickness = 0.5pt,
  forcex location ratio = 1,
  forcey location ratio = 0.5,  
  force length = 1cm,
  force line thickness = 1pt,
  force text = $f(t)$,
  display displacement = 1,
  display force = 1},
  vertical support line thickness/.estore in = \supltv,
  horizontal support line thickness/.estore in = \suplth,
  spring and damper line thickness/.estore in = \spdalth,
  mass line thickness/.estore in = \masslt,
  wheel line thickness/.estore in = \whlt,
  startx/.estore in = \stx,
  starty/.estore in = \sty,
  vertical support shade thickness/.estore in = \tx,
  horizontal support shade thickness/.estore in = \ty,
  support width/.estore in = \supw,
  support height/.estore in = \suph,
  mass height/.estore in = \massh,
  mass width/.estore in = \massw,
  massx/.estore in = \massx,  
  wheel radius/.estore in = \wrad,
  spring location ratio/.estore in = \sprlocr,
  spring ratio/.estore in = \str,
  damper width ratio/.estore in = \dwr,
  damper height ratio/.estore in = \dhr,
  damper piston size ratio/.estore in = \dr,
  damper piston position ratio/.estore in = \dpr,
  wheel location ratio/.estore in = \wwr,
  wheel size ratio/.estore in = \wradr,
  mass text/.estore in = \masstext,
  spring text/.estore in = \springtext,
  damper text/.estore in = \dampertext,
  disp line location ratio/.estore in = \displr,
  disp line/.estore in = \displ,
  disp arrow/.estore in = \dispar,
  disp text/.estore in = \disptext,
  disp line thickness/.estore in = \displth,
  forcex location ratio/.estore in = \forcexar,
  forcey location ratio/.estore in = \forceyar , 
  force length/.estore in = \forcel,
  force line thickness/.estore in = \forceth,
  force text/.estore in = \forcetext,
  display displacement/.estore in = \dispd,
  display force/.estore in = \dispf}

\newcommand{\sdofsys}[1][]{
\pgfkeys{/sdofsys, default, #1}
\tikzmath{
real \suplth, \supltv, \spdalth, \masslt, \whlt;
real \stx, \sty, \tx, \ty, \supw, \suph;
real \massh, \massx, \massw, \wrad, \wwr, \wradr, \sprlocr;
real \str, \dwr, \dhr, \dr, \dpr;
real \displ, \displr, \forcel, \forcexar, \forceyar, \forceth;
real \tempx, \tempy;
int \dispd, \dispf;
coordinate \c, \cs, \cd, \cde, \cmass, \cw, \cdisp, \cforce;
% Support shade and line coordinates
\tempx = \stx-\tx;     \tempy = \sty-\ty;      \c1 = (\tempx pt,\tempy pt);
\tempx = \stx-\tx;     \tempy = \sty+\suph;    \c2 = (\tempx pt,\tempy pt);
\tempx = \stx;         \tempy = \sty+\suph;    \c3 = (\tempx pt,\tempy pt);
\tempx = \stx;         \tempy = \sty;          \c4 = (\tempx pt,\tempy pt);
\tempx = \stx+\supw;   \tempy = \sty;          \c5 = (\tempx pt,\tempy pt);
\tempx = \stx+\supw;   \tempy = \sty-\ty;      \c6 = (\tempx pt,\tempy pt);
% Mass Coordinates
\tempx = \stx+\massx;       \tempy = \sty+\wrad*2;       \cmass1 = (\tempx pt,\tempy pt);
\tempx = \stx+\massx+\massw;  \tempy = \sty+\wrad*2+\massh;  \cmass2 = (\tempx pt,\tempy pt);
% Wheel Coordinates and Modified Wheel Radius
\tempx = \stx+\massx+\massw*\wwr;        \tempy = \sty+\wrad;     \cw1 = (\tempx pt,\tempy pt);  %Center of wheel 1
\tempx = \stx+\massx+\massw-\massw*\wwr; \tempy = \sty+\wrad;     \cw2 = (\tempx pt,\tempy pt);  %Center of wheel 2
\tempwr = \wrad-\whlt;
\newwr = \tempwr*\wradr;   % new radius
% Spring Coordinates
\spry = \sty+\wrad*2+\massh*(1-\sprlocr);  %spring y-axis;
\sf = \massx-\supltv/2-\masslt/2;          %total length of the spring
\st = \str*\sf;     %length of spring at the beginning
\sx1 = \stx+\supltv/2;      %spring x-coordinates;
\sx2 = \stx+\sf+\supltv/2+\masslt/2;
\cs1 = (\sx1 pt, \spry pt);
\cs2 = (\sx2 pt, \spry pt);
% Damper Coordinates
\dy = \sty+\wrad*2+\massh*\sprlocr;    %damper y-axis;
\df = \sf;                              %total length of damper
\dw = \dwr*\df;               %damper width;
\dh = \dhr*\massh;           %damper height
\dx1 = \stx+\supltv/2;
\dx2 = \stx+\supltv/2+(\df-\dw)/2;
\dx3 = \dx2+\dw*\dpr;
\dx4 = \dx2+\dw;
\dx5 = \stx+\df+\supltv/2+\masslt/2;
\cd1 = (\dx1 pt,\dy pt);
\cd2 = (\dx2 pt,\dy pt);
\cd3 = (\dx4 pt,\dy+\dh/2 pt);
\cd4 = (\dx2 pt,\dy+\dh/2 pt);
\cd5 = (\dx2 pt,\dy-\dh/2 pt);
\cd6 = (\dx4 pt,\dy-\dh/2 pt);
\cd7 = (\dx3 pt,\dy+\dh*\dr/2 pt);
\cd8 = (\dx3 pt,\dy-\dh*\dr/2 pt);
\cd9 = (\dx3 pt,\dy pt);
\cde = (\dx5 pt,\dy pt);
% Displacement Axis Coordinates
\tempx = \stx+\massx+\massw*\displr;  \tempy = \sty+\wrad*2+\massh;  \cdisp = (\tempx pt,\tempy pt);
%Force Arrow Coordinates
\tempx = \stx+\massx+\massw*\forcexar;  \tempy = \sty+\wrad*2+\massh*\forceyar;  \cforce = (\tempx pt,\tempy pt);
}

% Draw support shade and lines
\fill [gray] (\c1) -- (\c2) -- (\c3) -- (\c4) -- (\c5) -- (\c6) -- cycle;
\draw [line width = \suplth] (\c3) -- (\c4) [line width = \supltv] -- (\c5);

% Draw Spring
\draw [line width = \spdalth, decorate,
     decoration={zigzag,pre=lineto, pre length=\st pt, post=lineto, post length=\st*0.90 pt}]
     (\cs1) -- node[above] {\springtext} (\cs2);

%Draw Damper
\draw [line width = \spdalth] (\cd1) -- (\cd2);
\draw [line width = \spdalth] (\cd9) -- (\cde);
\draw [line width = \spdalth] (\cd3) -- (\cd4)  -- (\cd5)  -- node[below, yshift=0.05 cm] {\dampertext}(\cd6);
\draw [line width = \spdalth] (\cd7) -- (\cd8);

%Draw Mass
\draw [line width = \masslt] (\cmass1) rectangle (\cmass2) node[pos=0.5] {\masstext};

%Draw Wheels
\draw [line width = \whlt](\cw1) circle [radius=\newwr pt];
\draw [line width = \whlt](\cw2) circle [radius=\newwr pt];

\ifthenelse{\dispd=1}{
%Draw Displacement Vector
\draw [line width = \displth] (\cdisp) -- ++(0 cm, \displ);
\draw [line width = \displth] (\cdisp) ++(0 cm, \displ*0.8) [->] -- node[above, yshift=-0.1 cm]{\disptext} ++(\dispar, 0 cm);
}{}

\ifthenelse{\dispf=1}{
%Draw Force Vector
\draw [line width = \forceth] (\cforce) [->] -- node[above, yshift=-0.1 cm]{\forcetext} ++(\forcel, 0 cm);
}{}
}

%\begin{tikzpicture}
%\sdofsys[vertical support line thickness = 1pt,
%  horizontal support line thickness = 1pt,  
%  spring and damper line thickness = 1pt,
%  mass line thickness = 1pt,
%  wheel line thickness = 1pt,
%  startx = 0cm,
%  starty = 0cm,
%  vertical support shade thickness = 0.3cm,
%  horizontal support shade thickness = 0.3cm,    
%  support width = 4cm,
%  support height = 2cm,
%  mass height = 1cm,
%  mass width = 2cm,
%  massx = 1.5cm,
%  wheel radius = 0.15cm,
%  spring location ratio = 0.333,
%  spring ratio = 0.15,
%  damper width ratio = 0.15,
%  damper height ratio = 0.25,
%  damper piston size ratio = 0.8,
%  damper piston position ratio = 0.4,
%  wheel location ratio = 0.25,
%  wheel size ratio = 1.1]
%\end{tikzpicture}