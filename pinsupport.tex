
%%%%%%%%%%%%%%%%%%%%%%%%%%%%%%%%%%%%%%%%%
% Newcommand: pinsupport

\pgfkeys{
 /pinsupport/.is family, /pinsupport,
  default/.style = {triangle width = 0.5cm,
  line width = 1cm,
  line depth = 0.2cm,
  line thickness = 2pt},
  triangle width/.estore in = \triw,
  line width/.estore in = \linew,
  line depth/.estore in = \lined,
  line thickness/.estore in = \linet,
  x/.estore in = \x,
  y/.estore in = \y
}

\newcommand{\pinsupport}[1][]{
\pgfkeys{/pinsupport, default, #1}
\tikzmath{
real \triw, \linew, \lined, \x, \y;
coordinate \c1, \c2, \c3, \c4, \c5, \c6;
\r = \triw/125;
\c1 = (\x, \y);
\c2 = (\x-\triw/2, \y-\triw);
\c3 = (\x+\triw/2, \y-\triw);
\c4 = (\x-\linew/2, \y-\triw);
\c5 = (\x+\linew/2, \y-\triw);
\c6 = (\x-\linew/2, \y-\triw-\lined);
\c7 = (\x+\linew/2, \y-\triw-\lined);
}

\draw [line width = \linet] (\c1) -- (\c2) -- (\c3) -- cycle;
\filldraw [fill=white, line width = \linet] (\c1) circle[radius=\r cm];
\filldraw [gray] (\c6) rectangle (\c5);
\draw [line width = \linet] (\c4) -- (\c5);
}

% end pinsupport
%%%%%%%%%%%%%%%%%%%%%%%%%%%%%%%%%%%%%%%%%%%%%%%%


%\begin{tikzpicture}
%\pinsupport[triangle width=0.5cm, line width=1cm, line depth=0.2cm, line thickness = 2pt, x = 2cm, y = 2cm]
%\end{tikzpicture}
