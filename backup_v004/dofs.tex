%%%%%%%%%%%%%%%%%%%%%%%%%%%%%%%%%%%%%%%%%
% Newcommand: dofs

\newcommand\dofs[9]{%
    \def\xx{#1}   % start point x-coordinate
    \def\yy{#2}   % start point y-coordinate
    \def\dx{#3}   %dx for X-DOF
    \def\dy{#4}   %dy for Y-DOF
    \def\tdof{#5} %translational DOF offset
    \def\sa{#6}   %start angle of the rotational DOF
    \def\ea{#7}   %end angle of the rotational DOF
    \def\radi{#8} %radius of the rotational DOF
    \def\dofx{#9} %string for X-DOF
    \dofcontd
}

\newcommand\dofcontd[5]{%
\def\dofy{#1}; %string for Y-DOF
\def\dofz{#2}; %string for Rot-DOF
\def\xloc{#3};
\def\yloc{#4};
\def\rloc{#5};
\tikzmath{
coordinate \cjoint; % coordinate of the joint
\cjoint = (\xx,\yy);
}
\draw [->] (\cjoint) ++(\tdof cm,0) -- node[near end, \xloc]{\tiny $\dofx$} ++(\dx cm,0);
\draw [->] (\cjoint) ++(0, \tdof cm) -- node[near end, \yloc]{\tiny $\dofy$}++(0,\dy cm);
\draw [->] (\sa:\radi cm) ++(\cjoint) arc [start angle=\sa, end angle=\ea, radius=\radi cm] node[very near end, \rloc]{\tiny $\dofz$};
}
% end dofs
%%%%%%%%%%%%%%%%%%%%%%%%%%%%%%%%%%%%%%%%%%%%%%%%


%\begin{tikzpicture}
%\dofs{0}{0}{2}{3}{0.2}{110}{340}{0.5}{a}{2'}{3'}{right}{right}{right};
%\end{tikzpicture}